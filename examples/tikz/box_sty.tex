%%%
%draw input/output boxes
%used by record
%
%example:
%	\def\mar{v1,v2}
%
%	\RecordBox{ \mar }{1}{rec}
%
\tikzset{input/.style={anchor=north west}}
\tikzset{output/.style={anchor=north east}}
\tikzset{rbox/.style={rectangle,draw}}

\def\getlen#1{%
\pgfmathsetmacro{\lenarray}{0}% 
\foreach \i in #1{%
\pgfmathtruncatemacro{\lenarray}{\lenarray+1}% 
\global\let\lenarray\lenarray}%
}   
\NewDocumentCommand{\RecordBox}{m m m}{%var list,1=input/2=output, rec
	\getlen{ #1 }
	\begin{scope}[start chain=going below,node distance=-0.02cm]
	\foreach \v [count=\i] in #1 {
		\ifthenelse{\i=1}{
			\ifthenelse{#2=1}{
	    		\path let 
					\p1=($(#3.west)-(#3.east)$),
				 	\p2=($(#3.north)-(#3.south)$),
            	  	\n1 = {veclen(\p1)*0.16} 					%width
			  		,\n2 = {veclen(\p2)/\lenarray} 					%height
              		in node[on chain, rbox, input
			  		  ,minimum width=\n1
					  ,minimum height=\n2
					  ] (#3\v) at (#3.north west) {\v};
			}{
	    		\path let 
					\p1=($(#3.west)-(#3.east)$),
				 	\p2=($(#3.north)-(#3.south)$),
            	  	\n1 = {veclen(\p1)*0.16} 					%width
			  		,\n2 = {veclen(\p2)/\lenarray} 					%height
              		in node[on chain, rbox, output
			  		  ,minimum width=\n1
					  ,minimum height=\n2
					  ] (#3\v) at (#3.north east) {\v};

			}
		}{
    		\path let 
				\p1=($(#3.west)-(#3.east)$),
			  	\p2=($(#3.north)-(#3.south)$),
              	\n1 = {veclen(\p1)*0.16} 					%width
			  	,\n2 = {veclen(\p2)/\lenarray} 					%height
              	in node[on chain, rbox
			  		  ,minimum width=\n1
					  ,minimum height=\n2
					  ,anchor=north
					  ] (#3\v) {\v};
		}
	}
	\end{scope}
}
